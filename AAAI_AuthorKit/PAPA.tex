%File: formatting-instruction.tex
\documentclass[letterpaper]{article}
% AAAI format packages
\usepackage{aaai}
\usepackage{times}
\usepackage{helvet}
\usepackage{courier}
% Additional packages
\usepackage{amsmath}
\usepackage{amssymb}
\usepackage{amsthm}
\usepackage{algorithm}
\usepackage{algorithmic}
\usepackage{graphicx}
\usepackage{comment}
\newtheorem{defn}{Definition}
\newtheorem{lemma}{Lemma}
\newtheorem{thm}{Theorem}
\newtheorem{cor}{Corollary}
% END Additional packages
\frenchspacing
\setlength{\pdfpagewidth}{8.5in}
\setlength{\pdfpageheight}{11in}
\pdfinfo{
/Title (PAPA*: Path-Aware Parallel A*)
/Author (Aram Ebtekar, Mike Phillips, Sven Koenig, Maxim Likhachev)
/Keywords (weighted A* search, parallel algorithm, heuristic)
}
\setcounter{secnumdepth}{0}  
 \begin{document}
% The file aaai.sty is the style file for AAAI Press 
% proceedings, working notes, and technical reports.
%
\title{PAPA*: Path-Aware Parallel A*}
\author{Aram Ebtekar$^\dagger$ \and Mike Phillips$^\dagger$ \and Sven Koenig\thanks{University of Southern California, Los Angeles, CA 90089} \and Maxim Likhachev% <-this % stops a space
\thanks{Carnegie Mellon University, Pittsburgh, PA 15217}% <-this % stops a space
%
}
\author{AAAI 2015 Submission X}% anonymizer
\maketitle
\begin{abstract}
\begin{quote}
AAAI creates proceedings, working notes, and technical reports directly from electronic source furnished by the authors. To ensure that all papers in the publication have a uniform appearance, authors must adhere to the following instructions. 
\end{quote}
\end{abstract}

\section{Fancy Stuff}

Hello world.

\begin{algorithm}
\caption{$bound(s)$}
\label{alg:bound}
\begin{algorithmic}
\STATE $g_{front} := \infty$
\STATE $s' :=$ first node in $OPEN \cup BE$
\STATE $g_{back} := g(s) + f(s') - f(s) + (2\epsilon-w-1) c_l + 2\delta$
\WHILE{$g_{back} < g(s) \le g_{front}$}
\STATE $g_{front} := \min(g_{front},\;g_p(s') + \epsilon h(s',s) + 2\delta)$
\STATE $s' :=$ node following $s'$ in $OPEN \cup BE$
\STATE $g_{back} := g(s) + f(s') - f(s) + (2\epsilon-w-1) c_l + 2\delta$
\ENDWHILE
\RETURN $\min(g_{front},\;g_{back})$
\end{algorithmic}
\end{algorithm}

\begin{algorithm}
\caption{PAPA*}
\label{alg:PAPA*}
\begin{algorithmic}
\STATE LOCK
\WHILE{$g(s_{goal}) > bound(s_{goal})$}
\STATE remove an $s$ from $OPEN$ that has the smallest $f(s)$ among all states in $OPEN$ with $g(s) \le bound(s)$ and let $g_{bound} := bound(s)$
\IF{such an $s$ does not exist}
\STATE UNLOCK
\STATE wait until $OPEN$ or $BE$ change
\STATE LOCK
\STATE continue
\ENDIF
\STATE insert $s$ into $BE$
\STATE insert $s$ into $CLOSED$
\STATE UNLOCK
\STATE $S := getSucessors(s)$
\STATE LOCK
\FORALL{$s' \in S$}
\IF{$s'$ has not been generated yet}
\STATE $f(s') := g(s') := g_p(s') := \infty$
\ENDIF
\IF{$s' \notin CLOSED$}
\STATE $g_p(s') = \min(g_p(s),\; g_{bound} + \epsilon c(s,s'))$
\IF {$g(s') > g(s) + c(s,s')$}
\STATE $g(s') = g(s) + c(s,s')$
\STATE $f(s') = g(s') + wh(s')$
\STATE $bp(s') = s$
\STATE insert/update $s'$ in(to) $OPEN$ with key $f(s)$
\ENDIF
\ENDIF
\ENDFOR
\STATE remove $s$ from $BE$
\ENDWHILE
\end{algorithmic}
\end{algorithm}

Throughout this paper, we assume all edge costs are bounded below by $c_l$, that $\epsilon \ge 1$, $\delta \ge 0$ and $0 \le w \le \epsilon$. For most applications, we recommend using $\delta = 0$ and $w = \epsilon$. However, as we will see, it may pay to use $w < \epsilon$ when there are a lot of processors available, or $\delta > 0$ when the mean edge cost of paths is known to be much higher than the lower bound $c_l$.

\begin{defn}[TODO: these definitions are never used, so we don't need them]
We say a state $s$ is independent of $s'$ if $g(s) \le g_p(s') + \epsilon h(s',s) + 2\delta$. We say $s\in OPEN$ is safe to expand if $s$ is independent of all $s'\in OPEN \cup BE$.
\end{defn}

\begin{lemma}
\label{lem:indep}
At all times, for all states $s,s'$:
\[g(s) + f(s') - f(s) + (2\epsilon-w-1) c_l \le g_p(s') + \epsilon h(s',s).\]
\end{lemma}

\begin{proof}
\begin{eqnarray*}
&&g(s) + f(s') - f(s) + (2\epsilon-w-1) c_l
\\&=& g(s') + w(h(s') - h(s)) + (2\epsilon-w-1) c_l
\\&\le& g(s') + wh(s',s) + (2\epsilon-w-1) c_l
\\&\le& g(s') + \epsilon h(s',s) + (w-\epsilon) c_l + (2\epsilon-w-1) c_l
\\&=& g(s') + \epsilon h(s',s) + (\epsilon-1) c_l
\\&\le& g_p(s') + \epsilon h(s',s)
\end{eqnarray*}
\end{proof}

\begin{lemma}
\label{lem:subopt}
For all states $s$, $bound(s) \le \epsilon g^*(s) + k(s)\delta$ where $k(s)$ is the least number of edges used in a minimum-cost path to $s$. Hence, upon expanding $s$, $g(s) \le \epsilon g^*(s) + k(s)\delta$.
\end{lemma}

\begin{proof}
We proceed inductively by the order in which states are expanded. By construction, $bound(s)$ is bounded above by $g_p(s') + \epsilon h(s',s) + 2\delta$ for states $s'$ which are checked in the loop. As for the remaining states $s' \in OPEN \cup BE$, the algorithm ensures that $bound(s) \le g(s) + f(s') - f(s) + (2\epsilon-w-1) c_l + 2\delta$ for these by using a minimum representative. By Lemma \ref{lem:indep}, it follows that
\[bound(s) \le \min_{s' \in OPEN \cup BE} g_p(s') + \epsilon h(s',s) + 2\delta.\]

We complete the proof by showing that the latter expression is bounded above by $\epsilon g^*(s) + k(s)\delta$. Fix any optimal path to $s$, and let $s'$ be the first node on it which is in $OPEN \cup BE$. Let $s_p$ be the predecessor of $s'$ on this path. By the induction hypothesis, $g(s_p) \le \epsilon g^*(s_p) + k(s_p)\delta$. Therefore,
\begin{eqnarray*}
\epsilon g^*(s)&=& \epsilon\left(g^*(s_p) + c(s_p,s') + c^*(s',s)\right)
\\&\ge& g(s_p) -  k(s_p)\delta + \epsilon c(s_p,s') + \epsilon h(s',s)
\\&\ge& g_p(s') - k(s_p)\delta + \epsilon h(s',s)
\\&\ge& g_p(s') + \epsilon h(s',s) + 2\delta - k(s)\delta
\end{eqnarray*}
\end{proof}

TODO: clear up initial case where $s = s' = s_{start}$ and mention that use of $\delta$ was inspired by \cite{klein1997randomized}.

\begin{cor}
\label{cor:subopt}
If $\delta$ is not more than $m$ times the mean cost of the edges along the minimum-cost path to $s$, then upon expansion, $g(s) \le (\epsilon + m)g^*(s)$.
\end{cor}

\begin{proof}
We assumed $\delta \le m g^*(s) / k(s)$.
It follows from Lemma \ref{lem:subopt} that $g(s) \le \epsilon g^*(s) + k(s)\delta \le (\epsilon + m)g^*(s)$.
\end{proof}

\bibliographystyle{aaai}
\bibliography{PAPA}

\end{document}
