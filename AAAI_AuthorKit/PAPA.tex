%File: formatting-instruction.tex
\documentclass[letterpaper]{article}
% AAAI format packages
\usepackage{aaai}
\usepackage{times}
\usepackage{helvet}
\usepackage{courier}
% Additional packages
\usepackage{amsmath}
\usepackage{amssymb}
\usepackage{amsthm}
\usepackage{algorithm}
\usepackage{algorithmic}
\usepackage{graphicx}
\usepackage{comment}
\newtheorem{defn}{Definition}
\newtheorem{lemma}{Lemma}
\newtheorem{thm}{Theorem}
\newtheorem{cor}{Corollary}
\newtheorem{rul}{Expansion Rule}
% END Additional packages
\frenchspacing
\setlength{\pdfpagewidth}{8.5in}
\setlength{\pdfpageheight}{11in}
\pdfinfo{
/Title (PARA*: Parallel Anytime Repairing A*)
/Author (Aram Ebtekar, Mike Phillips, Sven Koenig, Maxim Likhachev)
/Keywords (weighted A* search, parallel algorithm, heuristic)
}
\setcounter{secnumdepth}{0}  
 \begin{document}
% The file aaai.sty is the style file for AAAI Press 
% proceedings, working notes, and technical reports.
%
\title{PARA*: Parallel Anytime Repairing A*}
\author{Aram Ebtekar$^\dagger$ \and Mike Phillips$^\dagger$ \and Sven Koenig\thanks{University of Southern California, Los Angeles, CA 90089} \and Maxim Likhachev% <-this % stops a space
\thanks{Carnegie Mellon University, Pittsburgh, PA 15217}% <-this % stops a space
%
}
\author{AAAI 2015 Submission X}% anonymizer
\maketitle
\begin{abstract}
\begin{quote}
PARA* is an anytime parallel heuristic search algorithm based on ARA* and PA*SE, which are in turn based on A*.
\end{quote}
\end{abstract}

\section{Introduction}

Breadth-first and depth-first search are generalized by a class of frontier-based search algorithms, differing mainly in the means by which nodes are selected from the frontier for expansion. In the weighted A* algorithm, the choice combines a greedy goal-directed bias to reduce search time, with a breadth-first bias which guarantees suboptimality by a specified factor. Anytime Reparing A* (ARA*) is an anytime search algorithm, gradually reducing the goal-directed bias to improve solution cost as much as planning time allows. With the advent of multi-core processors, making use of parallelism has become a priority for algorithm designers. Parallel A* for Slow Expansions (PA*SE) offers nearly linear speedup in the number of cores, provided the search is dominated by long expansion times.

In this paper, we present Parallel Anytime Repairing A* (PARA*), a simultaneous improvement over both ARA* and PA*SE. Like ARA*, it gradually reduces the weight which biases toward the goal, resulting in incrementally better solutions. Like PA*SE, it often offers approximately linear speedup. However, our design and analysis is tighter than PA*SE, even in the non-anytime setting. This enables several theoretical results, as well as moderate performance gains over PA*SE in certain settings.

Alternative selling point: our enhancements are two-fold. On one hand, we make algorithmic improvements which generalize the expansion rule, increasing parallelism while reducing the overhead for making use of said parallelism. On the other, we present a more detailed theoretical analysis, discussing the algorithm's properties in further detail.

\section{Problem Formulation}

We wish to find approximate single-pair shortest-paths. That is, given a directed graph with non-negative edge costs $c(s,s') \ge 0$, we must identify a path from $s_{start}$ to $s_{goal}$ whose cost is at most a specified factor $\epsilon\ge 1$ of the true distance $c^*(s_{start},s_{goal})$. We abbreviate $c^*(s_{start}, s)$ by $g^*(s)$.

We assume the distances can be estimated by a \textbf{consistent heuristic} $h$, meaning $h(s,s')\le c(s,s')$ and $h(s,s')\le h(s,s'') + h(s'',s')$ for all $s,s',s''$. Of course, consistency implies \textbf{admissibility}, meaning $h(s,s')\le c^*(s,s')$.

\section{Review of wPA*SE}

In order to clarify the role of the enhancements we will make, we present wPA*SE in an equivalent but slightly different form from the original in \cite{asdf}. Algorithm \ref{alg:parastar} is a skeleton for wPA*SE. It begins by clearing the data structures and expanding out all edges coming from the start node.

Intuitively, $OPEN$ represents the frontier of states which are candidates for expansion, initially consisting of the direct successors of $s_{start}$. Once selected for expansion, a state moves from $OPEN$ to $CLOSED$. $BE$ represents the freshly $CLOSED$ states, i.e. those which are still in the process of being expanded. Its cardinality $|BE|$ will never exceed the number of threads.

\begin{algorithm}
\caption{main()}
\label{alg:parastar}
\begin{algorithmic}
\STATE $OPEN := BE := CLOSED := FROZEN := \emptyset$
\STATE $g(s_{start}) := 0$
\STATE expand($s_{start}$)
\STATE run search() on multiple threads in parallel
\end{algorithmic}
\end{algorithm}

\begin{algorithm}
\caption{search()}
\label{alg:search}
\begin{algorithmic}
\WHILE{$g(s_{goal}) > bound(s_{goal})$}
\STATE among $s\in OPEN$ such that $g(s) \le bound(s)$, remove one with the smallest $f(s)$ and LOCK $s$
\IF{such an $s$ does not exist}
\STATE wait until $OPEN$ or $BE$ change
\STATE continue
\ENDIF
\STATE insert $s$ into $CLOSED$
\STATE insert $s$ into $BE$ with key $f(s)$
\STATE $v_{expand} := g(s)$
\STATE UNLOCK $s$
\STATE expand($s$)
\STATE $v(s) := v_{expand}$
\STATE remove $s$ from $BE$
\ENDWHILE
\end{algorithmic}
\end{algorithm}

\begin{algorithm}
\caption{expand($s$)}
\label{alg:expand}
\begin{algorithmic}
\FORALL{$s' \in successors(s)$}
\STATE LOCK $s'$
\IF{$s'$ has not been generated yet}
\STATE $g(s') := v(s') := \infty$
\ENDIF
\IF {$g(s') > g(s) + c(s,s')$}
\STATE $g(s') = g(s) + c(s,s')$
\STATE $bp(s') = s$
\IF{$s' \in CLOSED$}
\STATE insert $s'$ in $FROZEN$
\ELSE
\STATE insert/update $s'$ in $OPEN$ with key $f(s')$
\ENDIF
\ENDIF
\STATE UNLOCK $s'$
\ENDFOR
\end{algorithmic}
\end{algorithm}

Every thread of wPA*SE runs Algorithm \ref{alg:search} in parallel. $OPEN$ and $BE$ are represented by balanced binary trees sorted by $f(s) = g(s) + wh(s,s_{goal})$ for some parameter $w\ge 0$. Usually we recommend setting $w=\epsilon$, but alternatives are discussed later.

Each thread begins by attempting to extract an element $s\in OPEN$ which is \textbf{safe for expansion}. A state is considered safe if its current distance label $g(s)$ is known to be a good enough approximation of the true distance; that is, $g(s) \le \epsilon g^*(s)$. This condition is crucial since wPA*SE expands each node at most once. Each time a thread finds a safe $s$, it performs an expansion as described in Algorithm \ref{alg:expand}. The search terminates once the goal is safe for expansion. $bp$ stores back pointers which can be followed from any state $s$ back to $s_{start}$ to yield a path of costing at most $g(s)$.

The assignable variables $v(s)$ and the $FROZEN$ list are never used, and exist in the pseudocode only to aid the analysis. Intuitively, $v(s)$ is the distance label held by $s$ during its most recent expansion. If $g(s) < v(s)$, $s$ should be a candidate for future expansion. $FROZEN$ consists of nodes for which $g(s) < v(s)$, and hence would be candidates for expansion if not for the fact that $s$ was already expanded. Thus, $OPEN\cup FROZEN$ is precisely the set of states $s$ for which $g(s) < v(s)$. All other states have $g(s) = v(s)$.

There are many ways to define the auxiliary function $bound(s)$; as we will see, this contributes one of the major differences between wPA*SE and ePA*SE. In either case, it must satisfy $bound(s) \le \epsilon g^*(s)$, so that the condition $g(s) \le bound(s)$ suffices to ensure $s$ is safe for expansion. We begin with a simple example.

\begin{rul}[wA* rule]
A state $s\in OPEN$ is safe for expansion if its $f(s)$ value is minimal among states in $OPEN\cup BE$.
\end{rul}

Weighted A* always expands an $OPEN$ state with minimal $f(s)$, effectively requiring that $f(s')-f(s)\le 0$ where $s'$ is the first element (i.e. the one with minimal $f$-value) of $OPEN\cup BE$. If we implement $bound(s)$ as $g(s) + f(s') - f(s)$, the rule can be rewritten as requiring $g(s) \le bound(s)$. The wA* rule already offers a small degree of parallelism, as several states with the same minimum $f$-value can be expanded simultaneously. In order to generalize the safety criterion to increase parallelism, wPA*SE generalizes this rule.

\begin{rul}[wPA*SE rule]
A state $s\in OPEN$ is safe for expansion if its $g(s) \le bound(s)$ using the implementation of $bound$ listed in Algorithm \ref{alg:aux}.
\end{rul}

To explain the intuition behind $bound$, we argue that in order for $s\in OPEN$ to be unsafe for expansion, there must be an optimal path from $s$ passing through some $s'\in OPEN\cup BE$. Thus, $g^*(s) = g^*(s') + c^*(s',s) \ge g^*(s') + h(s',s)$.   It can be shown that, if some $s'$ makes $s$ unsafe, then there is such an $s'$ whose $g$-value is $\epsilon$-optimal. For $s$ to be safe, it then suffices that $g(s) \le g(s') + \epsilon h(s', s)$ since the latter quantity is at most $\epsilon (g(s') + h(s',s)) \le \epsilon g^*(s)$.

Provided $w \le \epsilon$, this inequality is guaranteed to hold when $f(s') \ge f(s)$. Hence, it suffices to check $s'$ for which $f(s') < f(s)$. See \cite{} for a proof that wPA*SE is $\max(w, \epsilon)$-optimal. In fact, it can be made $\epsilon$-optimal by checking all of $OPEN\cup BE$ instead of just the early elements. However, these checks can be expensive. Even in the former case, the cost of these checks for each state is at least proportional to the parallelism. The principal aim of our enhancements is to substantially reduce the number of checks needed while increasing parallelism.

Atomic locks are used for concurrency; for conceptual clarity, the mechanism presented here is considerably simpler than our C++11 implementation. We will not discuss details here, but it bears mentioning that every use of the main data structures is guarded by the same global lock.

\begin{algorithm}
\caption{Auxiliary Functions}
\label{alg:aux}
\begin{algorithmic}
\STATE \textbf{FUNCTION} $successors(s)$
\RETURN $\{s' \mid c(s,s')<\infty\}$

\STATE \textbf{FUNCTION} $f(s)$
\RETURN $g(s) + wh(s,s_{goal})$

\STATE \textbf{FUNCTION} $bound(s)$
\STATE $g_{front} := g(s)$
\STATE $s' :=$ first node in $OPEN \cup BE$
\WHILE{$f(s') < f(s)$ \AND $g(s) \le g_{front}$}
\STATE $g_{front} := \min(g_{front},\;g(s') + \epsilon h(s',s))$
\STATE $s' :=$ node following $s'$ in $OPEN \cup BE$
\ENDWHILE
\RETURN $g_{front}$
\end{algorithmic}
\end{algorithm}

\section{Improvements toward ePA*SE}

We introduce the variables $g_p(s)$. Their semantics are similar to $bound(s)$ but somewhat more intricate. While $bound(s)/\epsilon$ is a lower bound on $c^*(s_{start},s)$, $g_p(s)/\epsilon$ is a lower bound on the cost from $s_{start}$ to $s$, restricting ourselves to paths in which $s$ is immediately preceded by an expanded node. That is, $g_p(s) \le \epsilon (c^*(s_{start},s') + c(s',s))$ for all $s'\in CLOSED$. To maintain this invariant, we initialize $g_p(s)$ to $\infty$ just as we did for $g(s)$ and $v(s)$, and then add the following line immediately before the second \textbf{if} statement in expand($s$):
\[g_p(s') := \min(g_p(s'),\; g_{bound} + \epsilon c(s,s'))\]

Here, $g_{bound}$ is the lower bound on $\epsilon c^*(s_{start},s)$ computed when $bound(s)$ was called in the state expansion process, or $0$ if $s=s_{start}$. The enhanced version of $bound(s)$, listed in Algorithm \ref{alg:eaux}, makes use of a constant $c_l \ge 0$, denoting the best known lower bound on the graph's edge costs. $c_l$ can be 0 if we are agnostic about costs, but ePA*SE can make use of larger bounds if available.

\begin{rul}[ePA*SE rule]
A state $s\in OPEN$ is safe for expansion if its $g(s) \le bound(s)$ using the implementation of $bound$ listed in Algorithm \ref{alg:eaux}.
\end{rul}

TODO: explain changes to $bound(s)$.  Then prove completeness if finite out-degree, $w<\infty$ and $c_l > 0$

\section{PARA*: Parallel Anytime Reparing A*}

Finally, we note that by analogy with ARA*, ePA*SE can be made into an anytime algorithm, iteratively computing solutions with progressively smaller suboptimality bounds. In main(), instead of calling the parallel search() only once, it's called in a loop which terminates only when the agent decides it's no longer worthwhile to spend additional planning time to improve the solution. Between iterations, the thaw() procedure in Algorithm \ref{alg:eaux} must be called to place the $FROZEN$ states back into the $OPEN$ list.

The $g_p$ inequality should also hold for every $s'$ which was expanded (hence $CLOSED$) during prior anytime iterations. So the $g_p$ values need to be reset. thaw() already does this for the $OPEN$ list. When another state is seen for the first time in the present iteration (or equivalently, expand() encounters an $s'\notin OPEN\cup CLOSED$), it performs the reset operation
\[g_p(s') := g(s') + 2(\epsilon-1)c_l\]

This is a generalization of the $g_p(s') := \infty$ step from the non-anytime version.

\begin{algorithm}
\caption{Auxiliary Functions ePA*SE}
\label{alg:eaux}
\begin{algorithmic}
\STATE \textbf{FUNCTION} $g_{back}(s',s)$
\IF{$s' = NULL$}
\RETURN $\infty$
\ELSIF{$w \le \epsilon$}
\RETURN $g(s) + f(s') - f(s) + (2\epsilon-w-1)c_l$
\ELSE
\RETURN $\frac\epsilon w\left(g(s) + f(s') - f(s)\right) + (\epsilon-1)c_l$
\ENDIF

\STATE \textbf{FUNCTION} $bound(s)$
\STATE $g_{front} := g_p(s)$
\STATE $s' :=$ first node in $OPEN \cup BE$
\WHILE{$g_{back}(s',s) < g(s) \le g_{front}$}
\STATE $g_{front} := \min(g_{front},\;g_p(s') + \epsilon h(s',s))$
\STATE $s' :=$ node following $s'$ in $OPEN \cup BE$
\ENDWHILE
\RETURN $\min(g_{front},\;g_{back}(s',s))$

\STATE \textbf{PROCEDURE} $thaw()$
\STATE choose new $\epsilon \in [1,\infty]$ and $w \in [0,\infty]$
\STATE $OPEN := OPEN \cup FROZEN$ with keys $f(s)$
\STATE $CLOSED := FROZEN := \emptyset$
\FORALL{$s\in OPEN$}
\STATE $g_p(s) := g(s) + (\epsilon-1)\min(g(s),\;2c_l)$
\ENDFOR
\end{algorithmic}
\end{algorithm}

The following lemma lists some easily checked invariants of wPA*SE, ePA*SE and PARA*.

\begin{lemma}
\label{lem:prop}
At all times, the following invariants hold:
\begin{itemize}
\item $OPEN\cap CLOSED = \emptyset$
\item $BE\cup FROZEN \subseteq CLOSED$
\item $s\in OPEN\cup BE\cup FROZEN \Leftrightarrow g(s) < v(s)$
\item $s\notin OPEN\cup BE\cup FROZEN \Leftrightarrow g(s) = v(s)$
\item $g(bp(s)) + c(bp(s),s) \le g(s) \le min_{s'}\{v(s') + c(s',s)\}$
\item Following $bp(\cdot)$ from $s$ yields a path costing at most $g(s)$
\item $s\in OPEN\cup CLOSED \Rightarrow g(s) + (\epsilon-1)c_l \le g_p(s) \le \epsilon g(s)$
\item $s\in OPEN\cup CLOSED$ iff we had $g(s)<v(s)$ sometime during the current main() loop iteration
\end{itemize}
\end{lemma}

\begin{proof}
Induction on time.
\end{proof}

\section{Analysis}

In addition to making algorithmic enhancements, we present a deeper analysis than the wPA*SE paper \cite{}. We begin by looking at the properties of $bound$ from Algorithm \ref{alg:eaux}.

\begin{lemma}
\label{lem:indep}
At all times, for all states $s$ and $s'\notin \{s_{start},s\}$:
\[g_{back}(s',s) \le g_p(s') + \epsilon h(s',s).\]
\end{lemma}

\begin{proof}
If $w \le \epsilon$, then
\begin{eqnarray*}
&&g(s) + f(s') - f(s) + (2\epsilon-w-1)c_l
\\&=& g(s') + w(h(s',s_{goal}) - h(s,s_{goal})) + (2\epsilon-w-1)c_l
\\&\le& g(s') + wh(s',s) + (2\epsilon-w-1)c_l
\\&\le& g(s') + \epsilon h(s',s) + (w-\epsilon)c_l + (2\epsilon-w-1)c_l
\\&=& g(s') + (\epsilon-1)c_l + \epsilon h(s',s)
\\&\le& g_p(s') + \epsilon h(s',s)
\end{eqnarray*}
On the other hand, if $w > \epsilon$, then
\begin{eqnarray*}
&&\frac\epsilon w\left(g(s) + f(s') - f(s)\right) + (\epsilon-1)c_l
\\&=& \frac\epsilon w\left(g(s') + w(h(s',s_{goal}) - h(s,s_{goal})) \right) + (\epsilon-1)c_l
\\&\le& g(s') + \epsilon(h(s',s_{goal}) - h(s,s_{goal})) + (\epsilon-1)c_l
\\&\le& g(s') + (\epsilon-1)c_l + \epsilon h(s',s)
\\&\le& g_p(s') + \epsilon h(s',s)
\end{eqnarray*}
\end{proof}

\begin{lemma}
\label{lem:bound}
For all $s\in OPEN\cup BE$, $bound(s) \le \min_{s'\in OPEN \cup BE} g_p(s') + \epsilon h(s',s)$. Furthermore, $g(s) \le bound(s)$ iff $g(s) \le \min_{s'\in OPEN \cup BE} g_p(s') + \epsilon h(s',s)$.
\end{lemma}

\begin{proof}
By construction, $bound(s)$ is bounded above by $g_p(s') + \epsilon h(s',s)$ for $s'=s$ as well as for the other states $s'$ which are checked in the loop. As for the remaining states $s' \in OPEN \cup BE$, the algorithm ensures that $bound(s) \le g_{back}(s',s)$ for these by using a minimum representative. By Lemma \ref{lem:indep}, it follows that
\[bound(s) \le \min_{s' \in OPEN \cup BE} g_p(s') + \epsilon h(s',s).\]

To prove the second claim, note that the loop in $bound(s)$ terminates under only two conditions. Either $g(s) > g_{front}$, in which case we have $g(s) > g_p(s') + \epsilon h(s',s) \ge bound(s)$ for the $s'$ which began the final iteration; or $g(s) \le g_{back}(s',s)$, in which case $g(s) \le bound(s)$ iff $g(s) \le g_{front}$ iff $g(s) \le g_p(s') + \epsilon h(s',s)$ for all $s' \in OPEN \cup BE$.
\end{proof}

\begin{thm}
\label{thm:subopt}
For all $s\in OPEN\cup BE$, $bound(s) \le \epsilon g^*(s)$. Hence, for all $s\in CLOSED$, $g(s) \le v(s) \le \epsilon g^*(s)$.
\end{thm}

\begin{proof}
We proceed by induction on the order in which states are expanded.

Let $\pi = \langle s_0,s_1,\ldots,s_N \rangle$ be a minimum-cost path from $s_0 = s_{start}$ to $s_N = s\in OPEN\cup BE$. Choose the minimum $i$ such that $s_i\in OPEN\cup BE$. If $i = 1$, then
\[g_p(s_i) \le \epsilon g(s_i) = \epsilon g^*(s_i)\]
If $i \ge 2$, there are two cases to consider, depending on whether $s_{i-1}\in CLOSED$.

If so, then $expand(s_{i-1})$ has assigned to $g_p(s_i)$. Hence by the induction hypothesis,
\begin{eqnarray*}
g_p(s_i) &\le& v(s_{i-1}) + \epsilon c(s_{i-1},s_i)
\\&\le& \epsilon g^*(s_{i-1}) + \epsilon c(s_{i-1},s_i)
\\&=& \epsilon g^*(s_i)
\end{eqnarray*}

On the other hand, suppose $s_{i-1}\notin CLOSED$. Choose the maximum $j<i$ such that $s_j\in CLOSED$, or $j=0$ if there is no such $j$. Then $j\le i-2$ and, by the induction hypothesis, $g(s_j)\le \epsilon g^*(s_j)$. Furthermore, $g(s_k) = v(s_k)$ for all $j<k<i$. Let $g_{old}(s_i)$ denote the value of $g(s_i)$ at the start of the current main() loop iteration. Then,
\begin{eqnarray*}
g_p(s_i) &\le& g_{old}(s_i) + 2(\epsilon-1)c_l
\\&\le& v(s_j) + c^*(s_j,s_i) + 2(\epsilon-1)c_l
\\&\le& \epsilon g^*(s_j) + c^*(s_j,s_i) + 2(\epsilon-1)c_l
\\&=& \epsilon (g^*(s_j) + c^*(s_j,s_i)) + (\epsilon-1)(2c_l - c^*(s_j,s_i))
\\&\le& \epsilon g^*(s_i)
\end{eqnarray*}

In all three cases, we found that
\[g_p(s_i) + \epsilon h(s_i,s) \le \epsilon g^*(s_i) + \epsilon c^*(s_i,s) = \epsilon g^*(s).\]
Therefore, by Lemma \ref{lem:bound},
\[bound(s) \le \min_{s' \in OPEN \cup BE} g_p(s') + \epsilon h(s',s) \le \epsilon g^*(s).\]
\end{proof}

\begin{cor}
\label{cor:subopt}
At the end of a main() loop iteration, the path obtained by following the back-pointers $bp(\cdot)$ from $s_{goal}$ to $s_{start}$ is $\epsilon$-suboptimal.
\end{cor}

\begin{proof}
The termination condition of PARA* implies $g(s_{goal}) \le bound(s_{goal})$. By construction, the path given by following back-pointers costs at most $g(s_{goal})$. The claim now follows from Theorem \ref{thm:subopt}.
\end{proof}

\section{Performance Guarantees - Blind PARA*}

By deleting the while loop in bound($s$), we arrive at a simplified version of the algorithm which we call Blind PARA*. $g_p$ values are no longer used, so their computation can be ommitted. Blind PARA* can only expand states which would be proved safe in PARA* using zero iterations of the bound($s$) loop. Thus, every performance guarantees that we prove for Blind PARA* also holds for PARA*. 

\begin{thm}
\label{thm:depth}
If $w \le 1$, the parallel depth of Blind PARA* is bounded above by
\[\min\left(\frac{\epsilon g^*(s_{goal})}{(1-w)c_l},\;
\frac{\left(\epsilon g^*(s_{goal})\right)^2 }{(4\epsilon-2w-2)c_l^2}\right).\]
\end{thm}

\begin{proof}
We prove the two bounds separately. For the first, note that if the lowest $f$-value is $f_{min}$, every state with $f$-value up to $f_{min} + (2\epsilon-w-1)c_l$ can simultaneously be expanded. Since $h$ is consistent, the successors' $f$-values is at least $f_{min} + (1-w)c_l$. Therefore, the depth is at most
\[\frac{\epsilon g^*(s_{goal})}{(1-w)c_l}\]

For the other bound, write $t$ for $2\epsilon-w-1$. Notice that since $f$-values never decrease along paths, once the minimum $f$-value in $OPEN$ surpasses $f_{min}$, from then on all nodes with $f$-value up to $f_{min} + tc_l$ are always safe for expansion. And during each iteration of the simultaneous expansions, the $g$-value of all such nodes increases by at least $c_l$. Since $g$ cannot exceed $f$, this continues for at most $(f_{min} + tc_l) / c_l = f_{min}/c_l + t$ iterations, after which every node in $OPEN$ has $f$-value $\ge f_{min} + tc_l$. Continuing this process until $f_{min}$ exceeds $\epsilon g^*(s_{goal})$, a bound on the total iteration count is (TODO: fix this analysis)

\begin{eqnarray*}
&&t + 2t + 3t + ... + \epsilon g^*(s_{goal})/c_l
\\&\le& \epsilon  g^*(s_{goal})/(tc_l)(2+\epsilon g^*(s_{goal})/c_l+t)/2
\\&\le& (\epsilon g^*(s_{goal})/c_l )^2 (tc_l/(\epsilon g^*(s_{goal})) + 1/(2t) + c_l/(2\epsilon g^*(s_{goal})))
\\&\le& (\epsilon g^*(s_{goal})/c_l )^2 (t + 1/(2t) + 1/2)
\end{eqnarray*}
\end{proof}

\section{Edgewise Supobtimality}

Let $k(s)$ be the least number of edges used in a minimum-cost path to $s$ and fix $\delta > 0$. If $g_{front}$ and $g_{back}$ are each increased by $2\delta$, then by similar arguments to the proofs earlier in the paper, we find that, upon expanding $s$, $g(s) \le \epsilon g^*(s) + \delta k(s)$.

Here's an extension inspired by \cite{klein1997randomized}: suppose the mean edge cost $c_m$ along the optimal path is known to be much greater than the lower bound $c_l$. In such a case, the bound in Theorem \ref{thm:depth} scales poorly. To remedy the situation, we ``grow" the small edges, effectively running PARA* with $c_l' = c_l + \delta$ and $c'(s,s') = \max(c(s,s'), c_l')$.

\begin{thm}
\label{thm:delta}
If the mean cost of the edges along the minimum-cost path to $s$ is at least $c_m$, then upon expansion, $g(s) \le \epsilon(1+\delta/c_m)g^*(s)$. Therefore, to get the same optimality factor as $\epsilon$, we can set $\delta = (\epsilon-1)c_m$.
\end{thm}

\begin{proof}
We assumed $c_m \le g^*(s) / k(s)$, so $k(s) \le g^*(s) / c_m$.
It follows from Lemma \ref{thm:subopt} that $g'(s) \le \epsilon g'^*(s) \le \epsilon(g^*(s) + \delta k(s)) \le \epsilon(1+\delta/c_m)g^*(s)$.
\end{proof}

\begin{cor}
\label{cor:delta}
If $w \le 1$ and $c_m \le g^*(s) / k(s)$, the parallel depth of Blind PARA* can be improved to
\[\frac{\epsilon g^*(s_{goal})}{(1-w)(c_l+(\epsilon-1)c_m)}.\]
If in addition $c_m \ge g^*(s)/(mk(s))$, the depth is at most
\[\frac{\epsilon mk(s)}{(1-w)(\epsilon-1)}\]
\end{cor}

In other words, if we know the mean edge cost up to a small constant factor, we can find approximately optimal paths in a depth which is within a small factor of the ``omnicient" algorithm that expands only along the optimal path.

\section{Experiments}

\section{Conclusion}

\bibliographystyle{aaai}
\bibliography{PARA}

\end{document}
